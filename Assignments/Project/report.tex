\documentclass[10pt,a4paper, twocolumn]{article}
\usepackage[utf8]{inputenc}
\usepackage[T1]{fontenc}
\usepackage[english]{babel}
\usepackage[top=2cm, bottom=2cm, left=2cm, right=2cm]{geometry}
\usepackage{enumitem}					% For the margin in "itemize"
\usepackage{indentfirst}				% To indent first paragraph
\usepackage{tabularx}					% To have tables of full width
\usepackage[lofdepth,lotdepth]{subfig}	% For subfigures
\usepackage{amsmath,amssymb,amsthm,amsfonts}
\usepackage{sectsty}					% To configure sections fonts
\usepackage{multirow}					% For multirows in tables
\usepackage{makecell}					% To center in multirows
\usepackage{graphicx}
\usepackage{float}
\usepackage{cancel}

% Title declaration
\title{\LARGE \bf
EL2520 - Control Theory and Practice - Advanced \\
Project Lab : The Four Tank Process
}
\author{
Kartik Seshadri Chari \\
kartikc@kth.se \\
960807-0174 
\and
Devendra Sharma\\
devendra@kth.se\\
950815-5497
\and
Merav Modi\\
merav@kth.se\\
940120-}
\date{22 May,2019}

% Abstract configuration
\renewenvironment{abstract}{\bf \textit{Abstract} ---}

\begin{document}

% Configuration
\renewcommand{\thesection}{\Roman{section}.}		% change numerotation style
\renewcommand{\thesubsection}{\Alph{subsection}.}	% ...
\sectionfont{\normalfont}				% change section formating
\subsectionfont{\normalfont\itshape}	% change subsection formating
\setlist[itemize]{noitemsep, itemsep=2pt}
\setcounter{secnumdepth}{2}  % remove numerotation for subsubsection and above

% TITLE and ABSTRACT
\maketitle

\begin{abstract}
This document is a report for the Project Lab conducted under the EL2520 course. This project was divided into 2 distinct sessions. The first one involved deriving a physical model of a four-tank process for minimum phase and non-minimum phase configurations via experimentation and investigating the coupling between the tanks. Emphasis was also given to manually controlling the process to understand the performance limitations due to the non-minimum dynamics. The second instance was dedicated to testing the model-based decentralized PI controller and the robust Glover-McFarlane method.
\end{abstract}
% Section: MODELLING
\section{Modelling}
Here, the nonlinear differential equations describing the plant will be derived. Thus, the rate of change of volume in each tank is given as:
\begin{equation*}
A\dfrac{dh}{dt} = q\textsubscript{in} - q\textsubscript{out}
\end{equation*}
From Bernoulli's law, we have
\begin{equation*}
q\textsubscript{out} = a\sqrt{2gh} \ \ \  \forall g = 981 cm/s^2
\end{equation*}
Now, as there is a pump used, its flowrate is goverened as follows:
\begin{equation*}
q_L = \gamma ku, \ q_U = (1-\gamma) ku \ \ \forall \gamma \in [0,1]
\end{equation*}
$\forall \ q_L$ denotes the lower tank and $q_U$ denotes the upper tank
From the above equations, we can derive the non-linear system as follows:
\begin{align*}
A_1\dfrac{dh_1}{dt} &= -q_{out,1} + q_{out,3} + q_{L,1} \\
A_2\dfrac{dh_2}{dt} &= -q_{out,2} + q_{out,4} + q_{L,2} \\
A_3\dfrac{dh_3}{dt} &= -q_{out,3} + q_{U,2} \\
A_4\dfrac{dh_4}{dt} &= -q_{out,4} + q_{U,1}
\end{align*}
Thus, the final system of equations turns out to be:
\begin{equation}
\begin{aligned}
\dfrac{dh_1}{dt} &= \dfrac{-a_1}{A_1}\sqrt{2gh_1} + \dfrac{a_3}{A_1} \sqrt{2gh_3} + \dfrac{\gamma _1 k_1}{A_1}u_1\\
\dfrac{dh_2}{dt} &= \dfrac{-a_2}{A_2}\sqrt{2gh_2} + \dfrac{a_4}{A_2} \sqrt{2gh_4} + \dfrac{\gamma _2 k_2}{A_2}u_2\\
\dfrac{dh_3}{dt} &= \dfrac{-a_3}{A_3}\sqrt{2gh_3} + \dfrac{(1- \gamma _2) k_2}{A_3}u_2\\
\dfrac{dh_4}{dt} &= \dfrac{-a_4}{A_4}\sqrt{2gh_4} + \dfrac{(1- \gamma _1) k_1}{A_4}u_1
\end{aligned}
\label{eqn:nonlinear}
\end{equation}
\subsection{Equilibrium Equations}
For the equilibrium condition, we take the rate of change of height as 0 to get the following set of equations:
\begin{align*}
\dfrac{-a_1}{A_1}\sqrt{2gh_1^0} + \dfrac{a_3}{A_1} \sqrt{2gh_3^0} + \dfrac{\gamma _1 k_1}{A_1}u_1^0 &= 0\\
\dfrac{-a_2}{A_2}\sqrt{2gh_2^0} + \dfrac{a_4}{A_2} \sqrt{2gh_4^0} + \dfrac{\gamma _2 k_2}{A_2}u_2^0 &= 0\\
\dfrac{-a_3}{A_3}\sqrt{2gh_3^0} + \dfrac{(1- \gamma _2) k_2}{A_3}u_2^0 &= 0\\
\dfrac{-a_4}{A_4}\sqrt{2gh_4^0} + \dfrac{(1- \gamma _1) k_1}{A_4}u_1^0 &= 0 \\
y_i^0 = k_c h_i^0 \ \ \forall i = \{1,2,3,4\}
\end{align*}
where $h_i^0, \ u_i^0, \ y_i^0$ denote the steady state values.
\subsection{Linearization}
Let $\Delta u_i = u_i - u_i^0$, $\Delta h_i = h_i - h_i^0$ and $\Delta y_i = y_i - y_i^0$ denote the deviations from the equilibrium and also let \\ 
\[
u = 
\begin{bmatrix}
  \Delta u_1 \\
  \Delta u_2
\end{bmatrix}
,
x = 
\begin{bmatrix}
  \Delta h_1 \\
  \Delta h_2 \\
  \Delta h_3 \\
  \Delta h_4
\end{bmatrix}
,
y = 
\begin{bmatrix}
  \Delta y_1 \\
  \Delta y_2
\end{bmatrix}
\]
For Linearizing the above-derived system of non-linear equation \ref{eqn:nonlinear} around its equilibrium points, we use the Taylor Series of Expansion (neglecting HOTs) to obtain the following matrices:
\[
A =
\begin{bmatrix}
  \dfrac{\partial \Delta h_1}{\partial h_1} &
  \dfrac{\partial \Delta h_1}{\partial h_2} &
  \dfrac{\partial \Delta h_1}{\partial h_3} &
  \dfrac{\partial \Delta h_1}{\partial h_4} \\

  \dfrac{\partial \Delta h_2}{\partial h_1} &
  \dfrac{\partial \Delta h_2}{\partial h_2} &
  \dfrac{\partial \Delta h_2}{\partial h_3} &
  \dfrac{\partial \Delta h_2}{\partial h_4} \\

  \dfrac{\partial \Delta h_3}{\partial h_1} &
  \dfrac{\partial \Delta h_3}{\partial h_2} &
  \dfrac{\partial \Delta h_3}{\partial h_3} &
  \dfrac{\partial \Delta h_3}{\partial h_4} \\

  \dfrac{\partial \Delta h_4}{\partial h_1} &
  \dfrac{\partial \Delta h_4}{\partial h_2} &
  \dfrac{\partial \Delta h_4}{\partial h_3} &
  \dfrac{\partial \Delta h_4}{\partial h_4} \\
\end{bmatrix} \bigg |_{h_i^0, u_i^0}
\]

\[
B =
\begin{bmatrix}
  \dfrac{\partial \Delta h_1}{\partial u_1} &
  \dfrac{\partial \Delta h_1}{\partial u_2} \\

  \dfrac{\partial \Delta h_2}{\partial u_1} &
  \dfrac{\partial \Delta h_2}{\partial u_2} \\

  \dfrac{\partial \Delta h_3}{\partial u_1} &
  \dfrac{\partial \Delta h_3}{\partial u_2} \\

  \dfrac{\partial \Delta h_4}{\partial u_1} &
  \dfrac{\partial \Delta h_4}{\partial u_2} \\
\end{bmatrix} \bigg |_{h_i^0, u_i^0}
\]

\[
C =
\begin{bmatrix}
  \dfrac{\partial \Delta y_1}{\partial h_1} &
  \dfrac{\partial \Delta y_1}{\partial h_2} &
  \dfrac{\partial \Delta y_1}{\partial h_3} &
  \dfrac{\partial \Delta y_1}{\partial h_4} \\

  \dfrac{\partial \Delta y_2}{\partial h_1} &
  \dfrac{\partial \Delta y_2}{\partial h_2} &
  \dfrac{\partial \Delta y_2}{\partial h_3} &
  \dfrac{\partial \Delta y_2}{\partial h_4} \\
\end{bmatrix} \bigg |_{h_i^0, u_i^0}
\]

\[
D =
\begin{bmatrix}
  \dfrac{\partial \Delta y_1}{\partial u_1} &
  \dfrac{\partial \Delta y_1}{\partial u_2} \\

  \dfrac{\partial \Delta y_2}{\partial u_1} &
  \dfrac{\partial \Delta y_2}{\partial u_2} \\
\end{bmatrix} \bigg |_{h_i^0, u_i^0}
\]
After solving the above matrices, we get
\[
\dot{x} = Ax + Bu, \ \ \ \ \ 
\\
y = Cx + Du
\]
where,
\[
A =
\begin{bmatrix}
  -\dfrac{1}{T_1} & 0 & \dfrac{A_3}{A_1 T_3} & 0  \\
  0 & -\dfrac{1}{T_2} & 0 & \dfrac{A_4}{A_2 T_4}  \\
  0 & 0 & -\dfrac{1}{T_3} & 0 \\
  0 & 0 & 0 & -\dfrac{1}{T_4}
\end{bmatrix}
\]
\[
B =
\begin{bmatrix}
  \dfrac{\gamma_1 k_1}{A_1} & 0 \\
  0 & \dfrac{\gamma_2 k_2}{A_2} \\
  0 & \dfrac{(1-\gamma_2) k_2}{A_3} \\
  \dfrac{(1-\gamma_1) k_1}{A_4} & 0
\end{bmatrix}
\]
\[
C =
\begin{bmatrix}
k_c & 0 & 0 & 0 \\
0 & k_c & 0 & 0 \\
\end{bmatrix}
\]

and D = 0, with $T_i = \dfrac{A_i}{a_i}\sqrt{\dfrac{2h_i^0}{g}}$.

\subsection{Transfer Matrix}
In order to obtain the Transfer Matrix, we use the following formula - 
\begin{align*}
  G(s) &= C (sI - A)^{-1} B \\
\end{align*}
i.e.
G(s) = 
\[
\setlength{\arraycolsep}{1pt}
\begin{bmatrix}
  k_c & 0 & 0 & 0 \\
  0 & k_c & 0 & 0 
\end{bmatrix}
\setlength{\arraycolsep}{0.2pt}
\begin{bmatrix}
  \dfrac{T_1}{1 + s T_1} & 0 & \dfrac{\dfrac{A_3 T_1}{A_1}}{(1 + s T_1)(1 + s T_3)} & 0 \\
  0 & \dfrac{T_2}{1 + s T_2} & 0 & \dfrac{\dfrac{A_4 T_2}{A_2}}{(1+s T_2)(1 + s T_4)} \\
  0 & 0 & \dfrac{T_3}{1 + s T_3} & 0 \\
  0 & 0 & 0 & \dfrac{T_4}{1 + s T_4}
\end{bmatrix}
\]
\[
*\begin{bmatrix}
  \dfrac{\gamma_1 k_1}{A_1} & 0 \\
  0 & \dfrac{\gamma_2 k_2}{A_2} \\
  0 & \dfrac{(1-\gamma_2) k_2}{A_3} \\
  \dfrac{(1-\gamma_1) k_1}{A_4} & 0
\end{bmatrix} =
\]
\[
\begin{bmatrix}
  \dfrac{\gamma_1 k_1 c_1}{1 + s T_1} & \dfrac{(1-\gamma_2)k_2 c_1}{(1 + s T_3)(1 + s T_1)} \\\\
  \dfrac{(1-\gamma_1) k_1 c_2}{(1 + s T_4)(1 + s T_2)} & \dfrac{\gamma_2 k_2 c_2}{1 + s T_2}
\end{bmatrix}
\]
\subsection{Zeros of Transfer Matrix}
Zeros of the Transfer matrix G(s) are given by the numerator of the det(G(s)). In other words, zeros of the Transfer matrix can be obtained by solving the equation :
\begin{align*}
\gamma_1 \gamma_2 T_3 T_4 s^2 + \gamma_1 \gamma_2(T_3 + T_4) s + (\gamma_1 + \gamma_2 - 1) = 0 \\
\implies T_3 T_4 s^2 + (T_3 + T_4) s + \dfrac{\gamma_1 + \gamma_2 - 1}{\gamma_1 \gamma_2} = 0
\end{align*}
The solution of the above quadratic equation is:
\begin{align*}
  s_1 &= -\dfrac{T_3 + T_4}{2 T_3 T_4} + \dfrac{1}{2 T_3 T_4} \sqrt{(T_3 + T_4)^2 - 4 \dfrac{\gamma_1 + \gamma_2 - 1}{\gamma_1 \gamma_2}} \\
  s_2 &= -\dfrac{T_3 + T_4}{2 T_3 T_4} - \dfrac{1}{2 T_3 T_4} \sqrt{(T_3 + T_4)^2 - 4 \dfrac{\gamma_1 + \gamma_2 - 1}{\gamma_1 \gamma_2}} \\
\end{align*}
Here, as $s_2$ has both terms negative, the zero is on the Left Half of the s-plane. But when it comes to $s_1$, if the term $\dfrac{\gamma_1 + \gamma_2 - 1}{\gamma_1 \gamma_2}$ is positive, value of the square root term will be less than $T_3 + T_4$, which makes the zero negative. Thus, for the \textbf{Minimum Phase case}, $(\gamma_1 + \gamma_2 - 1) > 0$ i.e $\gamma_1 + \gamma_2 > 1$ Now as $\gamma_1,\gamma_2 \in [0,1]$, $\gamma_1 + \gamma_2 \leq 2$
\begin{equation*}
\therefore \ 1 < \gamma_1 + \gamma_2 \leq 2
\end{equation*}
For the \textbf{Non-minimum phase case}, one of the zeros is positive which means $(\gamma_1 + \gamma_2 - 1) < 0$. Also, as $\gamma_1,\gamma_2 \in [0,1]$, $\gamma_1 + \gamma_2 > 0$.
\begin{equation*}
\therefore \ 0 < \gamma_1 + \gamma_2 \leq 1
\end{equation*}
\subsection{RGA Analysis}
RGA(G(0)) = $G(0) .* G(0)\textsuperscript{-1}^T$
Thus,
\[
G(0) =
\begin{bmatrix}
  \gamma_1 k_1 c_1 & (1-\gamma_2)k_2 c_1 \\
  (1-\gamma_1) k_1 c_2 & \gamma_2 k_2 c_2
\end{bmatrix}
\]

\[
G(0)\textsuperscript{-1}^T =
(\dfrac{1}{\gamma_1 \gamma_2 k_1 k_2 c_1 c_2 - (1-\gamma_1) (1-\gamma_2) k_1 k_2 c_1 c_2})
\]
\[
\begin{bmatrix}
  \gamma_2 k_2 c_2 & -(1-\gamma_1) k_1 c_2 \\
  -(1-\gamma_2)k_2 c_1 & \gamma_1 k_1 c_1
\end{bmatrix}
\]

Hence, the RGA of $G(0)$ is given by

\begin{align*}
  RGA(G(0)) &= G(0) .* G(0)^{-T} \\
   &= 
  \begin{bmatrix}
	\dfrac{\gamma_1 \gamma_2}{\gamma_1 + \gamma_2 - 1}    
    & \dfrac{\gamma_1 + \gamma_2 - 1 - \gamma_1 \gamma_2}{\gamma_1 + \gamma_2 - 1} \\ \\
    \dfrac{\gamma_1 + \gamma_2 - 1 - \gamma_1 \gamma_2}{\gamma_1 + \gamma_2 - 1} & \dfrac{\gamma_1 \gamma_2}{\gamma_1 + \gamma_2 - 1}
  \end{bmatrix}
\end{align*}
Now, let $\lambda = \dfrac{\gamma_1 \gamma_2}{\gamma_1 + \gamma_2 - 1}$. Thus, we get
\[
RGA(G(0)) = 
\begin{bmatrix}
\lambda & 1-\lambda \\
1-\lambda & \lambda
\end{bmatrix}
\]
For \textbf{Minimum Phase case} : \\
$\lambda_1 = 0.625 = \lambda_2$. Thus, 
\[
RGA(G_{mp}(0) =
\begin{bmatrix}
  1.5625 & -0.5625 \\
  -0.5625 & 1.5625 \\
\end{bmatrix}
\] \\
For \textbf{Non-Minimum Phase case} : \\
$\lambda_1 = 0.375 = \lambda_2$. Thus, 
\[
RGA(G_{nmp}(0) =
\begin{bmatrix}
  -0.5625 & 1.5625 \\
  1.5625 & -0.5625 \\
\end{bmatrix}
\] 

\subsection{Determination of $k_1$ and $k_2$}
For determining the values of $k_1$ and $k_2$, we need to measure how fast the pump fills up a tank. Thus, for $k_1$, tank 1 was observed and the following experiment was conducted.
We stopped the outflow from tank 1 and ensured that only the pump input was allowed into the tank. This helped us obtain the value of $k_1$ using the following equation:
\begin{equation*}
\dfrac{dh_1}{dt} = \cancelto{0}{-\dfrac{a_1}{A_1}\sqrt{2gh_1}} + \cancelto{0}{\dfrac{a_3}{A_1}\sqrt{2gh_3}} + \dfrac{\gamma_1 k_1}{A_1}u_1 
\end{equation*}
We repeated the same process with the $tank_2$ for obtaining the value of $k_2$. \\
After multiple readings using different voltages, we obtained 
\begin{align*}
k_1 &= 4.4247 cm^3 / sV \\
k_2 &= 4.1092 cm^3 / sV
\end{align*}
\subsection{Determination of areas of holes}
Before proceeding with the procedure, some assumptions used are as follows:
\begin{itemize}
\item The cross-sectional areas of the tanks are the same i.e A = 15.52 $cm^2$
\item The areas $a_1$ and $a_2$ remain the same for both Minimum and non-minimum phase cases
\item The areas $a_3$ and $a_3$ will vary for Minimum and non-minimum phase cases
\end{itemize}
So, for calculating the areas, we first considered tanks 3 and 4 as they just had a single input and single output making the calculation for $a_3$ and $a_4$ simple. We let the upper 2 tanks settle at their equilibrium points $h_3^0$ and $h_4^0$ by keeping $u_1^0 = 7.5V$ and $u_2^0 = 7.5V$. 
\begin{align*}
\therefore \ a_3 &= \dfrac{(1- \gamma _2) k_2}{\sqrt{2gh_3^0}}u_2^0\\ 
and \ 
\ a_4 &= \dfrac{(1- \gamma _1) k_1}{\sqrt{2gh_4^0}}u_1^0
\end{align*}
Now, the entire system was driven to equilibrium and all the steady-state heights($h_i^0$) were measured. As we had 2 equations(one each for tank 1 and tank 2) and 2 unknown, the areas of holes 1 and 2 were calculated by solving the 2*2 linear system as follows:
\begin{align*}
a_1 &= \dfrac{a_3}{\sqrt{2gh_1^0}} \sqrt{2gh_3^0} + \dfrac{\gamma _1 k_1}{\sqrt{2gh_1^0}}u_1^0 \\
a_2 &= \dfrac{a_4}{\sqrt{2gh_2^0}} \sqrt{2gh_4^0} + \dfrac{\gamma _2 k_2}{\sqrt{2gh_2^0}}u_2^0 \\
\end{align*}
For \textbf{Minimum Phase Case}
\begin{align*}
a_1 &= 0.2250 \ cm^2\\
a_2 &= 0.2389 \ cm^2\\
a_3 &= 0.0633 \ cm^2\\
a_4 &= 0.0888 \ cm^2
\end{align*}
For \textbf{Non-Minimum Phase Case}
\begin{align*}
a_1 &= 0.2250 \ cm^2\\
a_2 &= 0.2389 \ cm^2\\
a_3 &= 0.115 \ cm^2\\
a_4 &= 0.213 \ cm^2
\end{align*}
\end{document}